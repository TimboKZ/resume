\documentclass{article}

\usepackage[sfdefault,light]{roboto}
\usepackage[T1]{fontenc}
\usepackage{setspace}
\usepackage{multicol}
\usepackage{fancyhdr}
\usepackage{xcolor}
\usepackage[a4paper, total={7.5in, 10.4in}]{geometry}
\usepackage{titling}
\usepackage{enumitem}
\setlist{leftmargin=7mm, labelsep=3mm}
\usepackage[colorlinks = true,
            linkcolor = blue,
            urlcolor  = blue,
            citecolor = blue,
            anchorcolor = blue]{hyperref}
\pagenumbering{gobble}
\renewcommand{\familydefault}{\sfdefault}
\setlength{\columnsep}{1cm}
\usepackage{titlesec}

\newcommand*{\TitleFont}{%
      \fontsize{44}{40}%
      \selectfont}

%\setlength{\droptitle}{-7em}

\fancypagestyle{references}{\fancyhf{}\renewcommand{\headrulewidth}{0pt}\fancyfoot[R]{References available on request}}

\titleformat{\section}
  {\normalfont\Large\bfseries}{\thesection}{0em}{}[{\color{blue}\titlerule[0.1pt]}]

\begin{document}

\begin{center}
   \huge{Timur Kuzhagaliyev}
\end{center}
\vspace*{\stretch{0.8}}

\thispagestyle{references}

\begin{multicols}{2}

\section*{\normalfont About Me}
I'm a computer science student at University College London, spending a year abroad in California Institute of Technology. I have a keen interest in software engineering, information theory, virtual \& augmented reality, robotics and applications of computer science in healthcare.

\begin{tabular}{ r l }
 \\ Contacts & Timur Kuzhagaliyev \\ 
 & \href{mailto:tim.kuzh@gmail.com}{tim.kuzh@gmail.com}\\
 & \href{tel:+44 7393 838116}{+44 7393 838116}\\
 & \href{skype:timur-gb}{timur-gb} on Skype\\\\
 Links & \href{https://github.com/TimboKZ}{https://github.com/TimboKZ}\\ 
 & GitHub page with the list of public projects\\\\
 & \href{https://foxypanda.me/}{https://foxypanda.me/}\\ 
 & Personal blog with some of the ideas\\
 & and information about recent projects\\
 \end{tabular}

\section*{\normalfont Education}

\begin{tabular}{ r l }
 2017 - 2018 & \textbf{California Institute of Technology} \\ 
 & \color{gray} Pasadena, California, USA\\
 \vspace{2mm}
 & Exchange programme, Computer Science\\
 2015 - 2019 & \textbf{University College London} \\ 
 & \color{gray} London, United Kingdom\\
 & MEng Computer Science\\
 \vspace{2mm}
 & Current Average: 85\%\\
 2013 - 2015 & \textbf{Mander Portman Woodward} \\
 & \color{gray} London, United Kingdom\\
 & \textit{A Levels}: Mathematics (A*), Chemistry (A*)\\
 \vspace{2mm}
 & Physics (A*), Further Mathematics (A*)\\
 2012 - 2013 & \textbf{Queen Ethelburga's College} \\
 & \color{gray} York, United Kingdom\\
 & \textit{GCSE}: Physics (A*), Chemistry (A*),\\
 & Further Mathematics (A*) and 6 others\\
\end{tabular}

\section*{\normalfont Work experience \emph{\small\color{gray}(expanded on the next page)}}

\textbf{UCL Surgical Robot Vision Research Group} \\
\emph{\color{gray}07/2017 to 09/2017 | HoloLens developer}
\vspace{2mm}
\\
\textbf{World Medical Education Social Enterprise} \\
\emph{\color{gray}07/2017 to 09/2017 | Android developer}
\vspace{2mm}
\\
\textbf{UCL Electrical Engineering department} \\
\emph{\color{gray}07/2017 to 09/2017 | Android developer}
\vspace{2mm}
\\
\textbf{Microsoft/UCL Institute of Child Health: Fizzyo} \\
\emph{\color{gray} 05/2017 to 08/2017 | technical lead, back-end engineer}
\vspace{2mm}
\\
\textbf{UCLH PEACH Reality} \\
\emph{\color{gray} 10/2016 to 04/2017 | technical lead, HoloLens developer}
\vspace{2mm}
\\
\textbf{ATM Energy UK Limited} \\
\emph{\color{gray} 04/2016 to 04/2017 | full stack developer, IT consultant}
\vspace{2mm}
\\
\textbf{UCL School of Life and Medical Sciences} \\
\emph{\color{gray} 01/2016 to 04/2016 | technical lead, full stack developer}

\columnbreak

\section*{\normalfont Hard skills \emph{\small\color{gray}(soft skills can be found on the next page)}}

\textbf{Web Design}: Experience with React, Angular 2, Vue, Sass, Less, LAMP and MEAN stacks, REST API development.
\vspace{2mm} \\
\textbf{Software Engineering}: Experience with continuous integration, requirements gathering, user experience design, agile development. Worked with Git, Travis CI, Linux (Ubuntu, Debian), Azure, AWS, Heroku, MySQL and Postgres.
\vspace{2mm} \\
\textbf{Computer Science}: Knowledge of complexity analysis, compiler theory, concurrency abstraction, formal logic, operating system design, information theory, distributed systems, linear algebra.
\vspace{2mm} \\
\textbf{Programming Languages}: Strong knowledge of JavaScript (incl. ES6 and Node.js), PHP, C\#, Java. Confident with C, Python, SQL, Bash scripting. Familiar with Go, Matlab, Haskell.
\vspace{2mm} \\
\textbf{Human Languages}: Fluent in Russian and English, beginner knowledge of Kazakh and Japanese.
\vspace{2mm} \\
\textbf{Digital Design}: Experience with most Adobe design tools, including strong knowledge of Photoshop, Premiere Pro and After Effects.

\section*{\normalfont Notable achievements}

\begin{itemize}
    \item Invited to 2018 SPIE Medical Imaging conference to present \textit{Augmented reality needle ablation guidance tool for irreversible electroporation in the pancreas} paper
    \item Worked on Fizzyo project, Tech4Good Digital Health 2017 award winner
    \item Featured on Microsoft HoloLens homepage with PEACH Reality project
	\item Microsoft Imagine Cup 2017 UK finalist with PEACH Reality project
    \item Awarded BCS/Microsoft/UCL Computer Science App Award for Excellence in 2016
    \item Awarded The MPW Outstanding Achievement Prize 2014/2015 for best A Level results in college
    \item Awarded Gold Certificate in UK Senior Mathematical Challenge in 2014
    \item Awarded 1 Year GCSE Progress Prize 2013 for outstanding GCSE results
\end{itemize}

\section*{\normalfont Extracurricular activities}

\begin{itemize}
    \item Elected as the Treasurer of UCLU Anime Society for 2016-2017
    \item Elected as the Armourer of UCL Kendo Club for 2016-2017
    \item Volunteered for IntoUniversity mentoring programme in 2015-2016
\end{itemize}

\vfill


\pagebreak

\section*{\normalfont Work experience \emph{\small\color{gray}(detailed)}}

\textbf{Surgical Robot Vision Research Group} \\
\emph{\color{gray}07/2017 to 09/2017 | HoloLens developer} \\
As a summer intern, I built a tool for irreversible electroporation (IRE) needle electrode guidance. I used HoloLens to display the holograms and visual guides, OptiTrack V120:Trio infrared tracker and optical markers for tracking, and an ultrasound machine to stream the ultrasound feed. I have also implemented a communication protocol for exchange and processing of tracking data and various calibration algorithms. Primary languages were C\# and Python.
\vspace{2mm}
\\
\textbf{World Medical Education Social Enterprise} \\
\emph{\color{gray}07/2017 to 09/2017 | Android developer} \\
I developed an Android app and a schema specification for a system of offline medical learning resources distribution using USB On-The-Go (OTG). Forward-compatible schema defines the structure of data on the USB device, while the app parses the schema to extract video paths, transcripts, atlas of definitions, encrypted assessment questions, etc.
\vspace{2mm}
\\
\textbf{UCL Electrical Engineering department} \\
\emph{\color{gray}07/2017 to 09/2017 | Android developer} \\
I developed mobile and desktop apps that interface with pressure and temperature sensors over Bluetooth. The system monitors drug delivery by analysing the temperature and pressure profiles in the canister before, during and after the patient inhales the drug.
\vspace{2mm}
\\
\textbf{Microsoft/UCL Institute of Child Health: Fizzyo} \\
\emph{\color{gray} 05/2017 to 08/2017 | technical lead, back-end engineer} \\
Fizzyo is meant to make collecting data and improving treatments for cystic fibrosis. As a summer intern, I supervised the development of all Fizzyo components, including mobile apps, web apps, backend, REST API, communication with HealthVault and games built on Unity. I have personally developed the backend for collection, parsing and analysis of medical trial data, based on a RESTful API. I used Node.js and PostgreSQL, with numerous libraries for unit and integration testing.
\vspace{2mm}
\\
\textbf{UCLH PEACH Reality} \\
\emph{\color{gray} 10/2016 to 04/2017 | technical lead, HoloLens developer} \\
I built a system for conversion of medical imaging data (CT, MRI scans) into 3D meshes with subsequent viewing, analysis and annotation of the meshes on Microsoft HoloLens. Trained neural network for segmentation of CT scans provided by InnerSight labs.
\vspace{2mm}
\\
\textbf{ATM Energy UK Limited} \\
\emph{\color{gray} 04/2016 to 04/2017 | full stack developer, IT consultant}\\
I performed general IT maintenance, web design and consulting for IT projects of the company.
\\\\
\textbf{UCL School of Life and Medical Sciences} \\
\emph{\color{gray} 01/2016 to 04/2016 | technical lead, full stack developer}
As a part of my industry placement, I built a tool for tracking of academic career progression of medical research staff on the LAMP stack.
\vfill

\columnbreak

\section*{\normalfont Soft skills}

\textbf{Leadership}: During \textit{all} of my group projects and industry placement experiences I took on the role of the team leader. I was responsible for either understanding client's requirements or producing my own, consequently explaining them to my team and eliminating any ambiguities. I was also responsible for distributing the workload and checking in on members of the team to make sure they can keep up, and take appropriate measures if there are problems. As a result, I've developed solid communication, coaching and management skills.
\vspace{2mm} \\
\textbf{Flexibility}: I have a deep passion for software engineering and problem solving, so I'm able to apply myself in multiple unrelated fields without losing concentration. I can take up many different roles in a project, while coaching my colleagues on relevant topics. This is best demonstrated by my experiences with Fizzyo (Microsoft) and Surgical Robot Vision lab, where I either had to supervise the development of multiple components of a system or develop multiple components myself.
\vspace{2mm} \\
\textbf{Communication}: By the virtue of student projects, industry placements, writing documentation for numerous open- and closed-source projects as well as university classes on professional communication, I have learnt to express my thoughts and ideas clearly and coherently. Often times this would involve adjusting the way I present an idea based on listener's background and technical literacy, which is a skill I picked up along the way. I'm often praised by my clients and colleagues for this particular skill.
\vspace{2mm} \\
\textbf{Time management}: I have been coding since I was 11 and had a reasonable amount of experience working on industry projects, so I can often spot potential pitfalls very early on in the development process. Having worked with multiple startups, I can also tell which features requested by the client are business-critical. Together, these 2 things help me prioritise certain requirements and schedule development accordingly. In group projects, to ensure that critical products are delivered on- or ahead of time, I always assign responsibilities based on person's technical proficiency and my personal level of confidence in their skills. If I work alone, I make sure to deliver the MVP as early as possible so that the client has something to evaluate and present to their customers.  
\vspace{2mm} \\
\textbf{Persuasion}: Thanks to my knowledge of computer science and software engineering, as well as experience working in related fields, I can often conceive optimal solutions to new problems on the fly. Important part of this is that I can always justify and argue why my solution is feasible, explaining everything step-by-step if needed. This has helped me stop my clients and peers from making decisions with dire consequences for their business. That said, when faced with a well-constructed argument, I will change my opinion.
\vfill

\end{multicols}

\end{document} 